\chapter{Related Work}
In this chapter, we briefly describe some streaming architectures applied to the Internet-of-Things scenario that are available in the literature.  

\paragraph{}
\cite{sensorsdatainthecloud} proposes an architecture called IoTCloud designed for real time robotics applications such as autonomous robot navigation. This architecture is structured over three layers:
\begin{itemize}
\item	 \emph{Gateway layer.} This layer is responsible for managing \emph{wireless sensor networks}. A gateway pushes sensor data to a message broker.
\item \emph{Publish-subscribe messaging layer.} The system supports different message brokers such as RabbitMQ \cite{rabbitmqonline}, Apache ActiveMQ\cite{apacheactivemqonline}, and Apache Kafka \cite{apachekafkaonline}. 
\item \emph{Cloud-based big data processing layer.} The platform uses Apache Storm as low-latency stream processing engine.
\end{itemize}

%IoTCloud really focuses on real time robotics applications. Cowbird, instead offers a general-purpose platform that is capable of sensing data from different type of sources.


\paragraph{}
\cite{lambdaarchitecturecosteffective} proposes a framework built in the Amazon cloud for processing large amounts of data coming from smart city sensors. This architecture, designed on top of EMR, uses Amazon Kinesis \cite{amazonkinesisonline} for data ingestion; Kinesis is also used, in combination with Amazon Lambda \cite{amazonlambdaonline}, for the speed processing layer. The batch layer is realized through MapReduce jobs executed on EMR. Amazon S3 is used as the persistent storage layer. 


\paragraph{}
\cite{alljoynlambda} brings the Lambda architecture to the AllJoyn framework \cite{alljoynframeworkonline}. Alljoyn is an open source project for the development of a universal software framework aimed at the interoperability among heterogeneous devices, dynamic creation of proximal networks and execution of distributed applications. However, AllJoyn does not scale well because it does not support communications among devices belonging to different broadcast domains (i.e., belonging to different subnets) and it does not provide any feature for the storage and real-time analytics of huge amount of data. \cite{alljoynlambda} overcomes this limitation integrating the AllJoyn framework with Apache Storm for real-time stream processing and MongoDB for the massive storage.

\paragraph{}
\cite{realtimeheterogeneousdata} proposes a framework for combining real-time data coming from smart farm sensors (i.e. temperature, humidity, amount of precipitation) with historical data. Historical data is stored in a MongoDB noSQL database while the data received from the sensor based systems is ingested through a message broker and processed by Apache Storm. They also propose a new scheduler implementation for Storm that employs a more efficient mechanism for using available resources in the cluster.

\paragraph{}
\cite{adaptiveiotapplications} describes a streaming based processing infrastructure for high throughput and low latency IoT real-time analytics services. They propose a data adaptive mechanism for heterogeneous data stream integration. In particular, they developed a framework for schema description, flow based processing definition, and user defined operators. These mechanisms allow to build a common IoT service for all kinds of real-time analytic applications that use different types of IoT data formats. They implemented the system using Apache Spark streaming. IoT events are captured by Kafka and then ingested into the Spark cluster through it.

\paragraph{}
From an architectural perspective, all the above described systems, are applications of the Lambda architecture. The above mentioned works focus on immediately ingesting the generated sensor data into stream processing technologies. In our work, we extended this concept bringing part of the computation close to the data generation. Unlike all the aforementioned systems, our architecture does not use the traditional concept of Lambda or streaming architecture but for some type of SWAN-Song expressions we perform the evaluation where the data is generated. Computing the sensor data close to their sources reduces extra-overheads induced by transmission to a cluster system or by fault-tolerant mechanisms offered by distributed message brokers. 

Our approach is made possible by the design of Cowbird. In fact, the Cowbird implementation uses polling threads to fetch sensors data into the system. Hence, part of the computation can be accomplished by the systems responsible for polling the data when it is possible. The design of Cowbird makes it possible to poll basically any type of data from any web resource. This is also another substantial difference with other available IoT streaming architecture implementations that are more specialized on particular scenarios.





